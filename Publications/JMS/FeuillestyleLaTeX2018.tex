% !TEX encoding = IsoLatin
\documentclass[12pt, a4paper]{article}
\usepackage[utf8]{inputenc}
\usepackage[T1]{fontenc}
%\usepackage[latin1]{inputenc}
\usepackage[hmargin = 25mm, vmargin = 25mm]{geometry}
%\geometry{letterpaper}                   % ... or a4paper or a5paper or ... 
%\geometry{landscape}                % Activate for for rotated page geometry
%\usepackage[parfill]{parskip}    % Activate to begin paragraphs with an empty line rather than an indent
\usepackage{graphicx}
\usepackage{amssymb, amsmath, amsthm}
\usepackage{epstopdf}
\usepackage{moreverb}
\usepackage{hyperref}
\usepackage{algorithm,algorithmic}
\usepackage[french]{babel}
\usepackage{hyperref}
\usepackage{fancyhdr} 
\pagestyle{fancy}

\renewcommand{\headrulewidth}{0pt}
\fancyhead[C]{} 
\fancyhead[L]{}
\fancyhead[R]{}

\renewcommand{\footrulewidth}{0pt}
\fancyfoot[C]{\footnotesize{$13^{es}$ Journées de méthodologie statistique de l’Insee (JMS) / 12-14 juin 2018 / PARIS}}
\fancyfoot[R]{\thepage}

\begin{document}

\begin{center}
\includegraphics[width=15cm]{head_jms.png} 
\line(1,0){450}
\vspace{5mm}
\textbf{{\huge E}\Large STIMATION SUR PETITS DOMAINES PAR SCISSION DES POIDS}
\end{center}

\begin{center}
\textit{Toky RANDRIANASOLO(*)(**), Yves TILL\'E(**), Jimmy ARMOOGUM(*)} \\
%- N. B. le premier nom doit être celui de l'auteur qui effectuera la présentation orale (sauf contributions associées)
\vspace{2mm}
\textit{(*)IFSTTAR, Département \'Economie et Sociologie des Transports}\\ 
\textit{(**) Université de Neuchâtel} \\
\vspace{2mm}
\url{marc.christine@insee.fr} 
\end{center}
\vspace{5mm}
\small{{\bf Mots-cl\'es.} \'Echantillonnage, Calcul de répartition, Optimisation, Dispersion des poids, Répartition de Neyman.}

\begin{center}
\line(1,0){450}
\end{center}


\section*{Résumé en $350$ mots}
erger
ergqer

\section*{Abstract en $5$ à $10$ lignes}

zerz
zefzr



\section*{Introduction}

Texte
dfrgwdgwdfgqegrhdrtudr

Texte

Texte

\section{Titre de niveau 1 numéroté}

Texte

Texte

Texte

\subsection{Titre de niveau 2 numéroté}

Texte

Texte

Un exemple de note de bas de page\footnote{Note de bas de page Note de bas de page Note de bas de page Note de bas de page Note de bas de page Note de bas de page Note de bas de page Note de bas de page Note de bas de page Note de bas de
page Note de bas de page.}.

\subsubsection{Titre de niveau 3 numéroté}

Texte

Texte

Texte


\section{Titre de niveau 1 numéroté}

Texte

Texte

Texte

\subsection{Titre de niveau 2 numéroté}

Texte

Texte

Texte

\nocite{*}
%\bibliographystyle{acm}
%\bibliography{DB}

\end{document}  