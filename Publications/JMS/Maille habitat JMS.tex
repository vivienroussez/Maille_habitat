% !TEX encoding = IsoLatin
\documentclass[12pt, a4paper]{article}
\usepackage[utf8]{inputenc}
\usepackage[T1]{fontenc}
%\usepackage[latin1]{inputenc}
\usepackage[hmargin = 25mm, vmargin = 25mm]{geometry}
%\geometry{letterpaper}                   % ... or a4paper or a5paper or ... 
%\geometry{landscape}                % Activate for for rotated page geometry
%\usepackage[parfill]{parskip}    % Activate to begin paragraphs with an empty line rather than an indent
\usepackage{graphicx}
\usepackage{amssymb, amsmath, amsthm}
\usepackage{epstopdf}
\usepackage{moreverb}
\usepackage{hyperref}
\usepackage{algorithm,algorithmic}
\usepackage[french]{babel}
\usepackage{hyperref}
\usepackage{fancyhdr} 
\pagestyle{fancy}

\renewcommand{\headrulewidth}{0pt}
\fancyhead[C]{} 
\fancyhead[L]{}
\fancyhead[R]{}

\renewcommand{\footrulewidth}{0pt}
\fancyfoot[C]{\footnotesize{$13^{es}$ Journées de méthodologie statistique de l’Insee (JMS) / 12-14 juin 2018 / PARIS}}
\fancyfoot[R]{\thepage}

\begin{document}

\begin{center}
\includegraphics[width=15cm]{head_jms2018.png} 
\line(1,0){450}
\vspace{5mm}
\textbf{{\huge \'E}\Large LABORATION D'UNE MAILLE GEOGRAPHIQUE POUR L'HABITAT}
\end{center}

\begin{center}
\textit{Solène COLIN(*), Vivien ROUSSEZ(*)} \\
%- N. B. le premier nom doit être celui de l'auteur qui effectuera la présentation orale (sauf contributions associées)
\vspace{2mm}
\textit{(*) CGDD, Service de la Donnée et des \'Etudes Statistiques}\\ 

\vspace{2mm}
\url{solene-c.colin@developpement-durable.gouv.fr} \url{vivien.roussez@developpement-durable.gouv.fr} 
\end{center}
\vspace{5mm}
\small{{\bf Mots-cl\'es.} Analyse spatiale, graph mining, clustering, analyse multidimensionnelle, cartographie.}

\begin{center}
\line(1,0){450}
\end{center}


\section*{Résumé}

Le SDES a lancé en 2017 un projet visant à construire un maillage du territoire à même de rendre compte des disparités territoriales sur les enjeux propres au logement. En effet, ni les échelles administratives, ni les zonages d'études de l'Insee (zones d'emploi, bassins de vie) ne sont adaptées pour l'analyse localisée du logement, car ils mêlent dans les mêmes mailles des types d'habitat différents (urbain et périurbains notamment). Pour cela, neuf indicateurs représentatifs de l'équilibre des marchés du logement ont été sélectionné pour alimenter la méthode de \textbf{régionalisation}, qui regroupe les communes en bassins homogènes. La méthode retenue est l'algorithme \textbf{SKATER} (Spatial Klustering Analysis by Tree Edge Removal), qui s'appuie sur l'exploration de graphe et notamment la notion d'arbre portant minimal. \\

Différentes simulations ont été menées pour déterminer la taille de ces mailles, et les résultats ont fait l'objet de tests au niveau régional. Ces mailles mettent au premier plan les disparités propres au logement (pouvoir d'achat immobilier, taille des ménages), alors que la maille communale fait principalement ressortir les disparités liées au degré d'urbanité. Elle permet également de distinguer les villes-centre de leur périphérie, et donc d'isoler les enjeux propres à ces espaces très différents sur le plan du logement. \\

L'analyse de ces mailles permet de mettre en exergue six types de marchés locaux du logement, principalement distingués par le niveau des prix, la composition des ménages (leur taille) et du parc de logement. Au fil du temps, les spécificités de ces marchés se renforcent et ceux déjà en tension voient leurs déséquilibres se renforcer. A l'inverse, le ralentissement démographique et le vieillissement à l'\oe uvre dans les espaces faiblement peuplés contribuent à la faible dynamique des marchés où la demande est déjà modérée. Par ailleurs, ces marchés présentent un degré inégal d'homogénéité, et on trouve d'avantage de diversité entre les communes de marchés de l'Est de la France.

\section*{Abstract}

Partant du constat que les mailles géographiques usuelles ne permettent pas une analyse fine des marchés du logement, le SDES a élaboré une maille ad hoc. Pour cela, 9 indicateurs emblématiques de la demande et de l'offre des marchés locaux du logement ont été sélectionnés et ont alimenté une méthode de \textbf{régionalisation} utilisant la théorie des graphes et les arbres portants minimaux. Ce maillage permet une lecture facilitée des dynamiques territoriales sur les plans du logement et de la démographie.


\section*{Introduction}

Les mailles habitat constuites par le SDES ont pour objectif de constituer une échelle pertinente pour l'observation et l'analyse des enjeux territoriaux liés à l'habitat. Elle permet de répondre à trois objectifs :

\begin{itemize}
\item Lisser visuellement l'information pour donner des cartes lisibles
\item Conserver au maximum les disparités territoriales, tout en faisant ressortir les enjeux propres à l'habitat
\item Alimenter la connaissance et les diagnostics locaux
\end{itemize}


Pour cela, le SDES a construit des ensembles de communes qui \emph{se ressemblent} sur le domaine de l'habitat. Cette approche diffère de celle de l'Insee qui, pour ses zonages d'études, regroupe les communes qui sont fortement connectées les unes aux autres. Ce degré de connexion s'apprécie à l'aune du nombre de navettes domicile-travail entre les communes (pour les zones d'emploi) et les flux (théoriques) d'habitant se déplaçant de leur commune de résidence vers le pôle de services le plus proche (pour les bassins de vie). Pour plus d'information sur les maillages produits par l'Insee, consultez \href{https://www.insee.fr/fr/information/2114631}{cette page}.

\section{Sélectionner des indicateurs pertinents}

\subsection{Quels indicateurs}




\subsection{Cartographie des indicateurs}

\subsection{Validation des indicateurs}

\subsubsection{Analyse en composantes principales}

\subsubsection{Classification communale}



\section{La régionalisation}

La \textbf{régionalisation} désigne l'opération qui consiste à regrouper des unités géographiques élémentaires en un ensemble \textbf{contigu}, selon des critères statistiques. Il existe un grand nombre de méthodes de régionalisation, dont une partie est comparée dans \href{http://journals.sagepub.com/doi/pdf/10.1177/0160017607301605}{cet article}. Le SDES a retenu l'algorithme SKATER (Spatial Klustering Analysis by Tree Edge Removal), implémenté et mis à disposition dans le package \textsc{spdep} du logiciel statistique libre\textsc{R}.

\subsection{L'algorithme SKATER}

Cet algorithme fonctionne en 4 étapes :

\begin{enumerate}
\item Construction de la matrice de contiguïté $\rightarrow$ obtention d'un \emph{graphe} (un noeud = une commune et un lien = relation de contiguïté entre deux communes)
\item Pondération de ce graphe avec les dissimilarités calculées à partir des indicateurs (distance euclidienne)
\item Construction de l'\textbf{arbre} portant minimal, en retenant le lien avec le voisin le plus ressemblant pour chaque n\oe ud du graphe
\item Suppression itérative des branches de l'arbre maximisant la variance inter-classes des sous-graphes obtenus après élagage.
\end{enumerate}

Il peut donc être vu comme une forme de classification descendante hiérarchique opérée sur les liens de l'arbre portant minimal. Le fait de travailler à partir de cet arbre portant minimal plutôt que sur les n\oe uds du graphe garantit la contiguïté des mailles finales.

\subsection{Découper le problème}

\subsection{Les différentes simulations}



\section{Caractérisation des mailles}

\subsection{L'habitat au premier plan}

\subsection{Typologie des mailles}

\subsection{Cohésion et dynamique de ces mailles}

\nocite{*}
%\bibliographystyle{acm}
%\bibliography{DB}

\end{document}  